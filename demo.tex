\documentclass[10pt]{beamer}
\usetheme[progressbar=frametitle]{metropolis}
\usepackage{amsmath}
\usepackage{amsfonts}
\usepackage{commath}
\usepackage{mathtools}
\usepackage{tabu}
\usepackage{booktabs}
\usepackage{bm}
\usepackage{xfrac}
\usepackage{listings}
\newcommand{\RR}{\mathbb{R}}
\DeclarePairedDelimiter\ip{\langle }{\rangle}
\DeclareMathOperator{\proj}{proj}

\title{Importance Sampling in Ray Tracing}
\subtitle{}
\date{\today}
\author{Jonathan Hayase \and Anqi He}
\institute{Math 164 -- Scientific Computing -- FA18}

\begin{document}

\maketitle

\begin{frame}{Table of contents}
  \setbeamertemplate{section in toc}[sections numbered]
  \tableofcontents[hideallsubsections]
\end{frame}

\begin{frame}{Introduction}
  \begin{itemize}
  \item Reproducing the macroscopic behavior of light is one of the fundamental problem domains in computer graphics.
  \item Simplifying assumptions are necessary to make computation tractable.
  \item Raytracing seeks to simulate light by modeling photons as particles which propagate along straight lines.
  \item Thus, we are only concerned with what happens when light ``bounces''.
  \end{itemize}
\end{frame}

\section{Theory}


\begin{frame}{The Rendering Equation}
  \[L_{o}(\mathbf x, \bm {\omega_{o}}, \lambda) = L_{e}(\mathbf x, \bm{\omega_o}, \lambda) + \int_\Omega f_s(\mathbf x, \bm{\omega_i}, \bm{\omega_o}, \lambda)L_i(\mathbf x, \bm{\omega_i}, \lambda)\ip{\bm n, \bm{\omega_i}} \dif \bm{\omega_i}.\]

  \hrulefill

  \begin{center}
    \begin{tabu}{clcl}
      \(L_{o}\)& outbound radiance & \(\bm{\omega_o}\)& outbound radiance direction\\
      \(L_{i}\)& inbound radiance & \(\bm{\omega_i}\) &inbound radiance direction\\
      \(L_{e}\)& emission radiance & \(\mathbf x\) &location in space\\
      \(f_s\) & scattering distribution & \(\lambda\)& spectral wavelength\\
       \(\bm n\)& surface normal at \(\mathbf x\) & \(\Omega\) & unit hemisphere above \(\bm n\)
    \end{tabu}
  \end{center}
\end{frame}
\end{document}
